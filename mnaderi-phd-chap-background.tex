%#######################################################################################################################
% PhD Thesis
% Department of Computer Science
% University College London
%
% Mohsen Naderi (mohsen@mnaderi.com)
%#######################################################################################################################


%#######################################################################################################################
\chapter{Background}
\label{chap:background}
%#######################################################################################################################


%=======================================================================================================================
\section{Introduction}
%=======================================================================================================================

A flash crash is a sudden fall in financial market prices followed by a quick recovery.
The best-known incident of this kind, which made the term \textit{flash crash} popular, happened on 6~May~2010.
On that day U.S. equities and futures markets collapsed and recovered to almost the same levels in a short period of
time.

%=======================================================================================================================
\section{May 2010 Flash Crash}
\label{sec:back:flash-crash-2010}
%=======================================================================================================================


%=======================================================================================================================
\section{Summary}
\label{sec:back:summary}
%=======================================================================================================================

\begin{mynotes}
These are just notes
\end{mynotes}

This chapter first reviewed the flash crash of 6~May~2010 as well as a number of other market disruptions that have many properties in common with the flash crash.